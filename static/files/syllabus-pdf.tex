\documentclass{article}
\usepackage{booktabs, graphicx, hyperref, fontspec}
\usepackage{sectsty}
\allsectionsfont{\sffamily}
\usepackage[margin=1in]{geometry}
\hypersetup{
  colorlinks = true,
  urlcolor = cyan,
 }
 \providecommand{\tightlist}{%
  \setlength{\itemsep}{0pt}\setlength{\parskip}{0pt}}
\newcommand*{\authorfont}{\fontfamily{phv}\selectfont}
\usepackage[]{Fira Sans}

\begin{document}

\sffamily

\centerline{\Huge History of Economic Thought}

\vspace{3 mm}

\centerline{\large Dr.~Ryan Safner}
\vspace{2 mm}
\centerline{\large \href{http://thoughtF20.classes.ryansafner.com}{thoughtF20.classes.ryansafner.com}}

\vspace{5 mm}

\begin{tabular}{@{}p{3.5in}p{3.5in}}           
\textbf{Course}: ECON 452  & \textbf{Email:}  \href{mailto:safner@hood.edu}{\nolinkurl{safner@hood.edu}}\\
\textbf{Room}: ON ZOOM & \textbf{Office:}  ON ZOOM\\
\textbf{Meets}: MW 3:25 P.M--4:40 P.M. & \textbf{Hours:} TuTh 10:30--11:30 A.M.\\ 
\end{tabular}

\vspace{5 mm}

\hrule


\begin{quote}
``Economics is what economists do\ldots{}'' - Jacob Viner
\end{quote}

\begin{quote}
``\ldots and economists do economics.'' - Frank Knight
\end{quote}

\begin{quote}
Quoted ironically in James M. Buchanan, 1964, ``What Should Economists
Do?''
\end{quote}

\textbf{The History of Economic Thought} explores the development of
economic principles and tools and the major economic thinkers to explain
why the discipline and practice of economics looks the way it does
today. By studying principles and intermediate micro- and
macro-economics, you may get the impression that economics is a complete
system of agreed-upon scientific laws to be applied individuals or the
economy for understanding and recommending policy. This course will
strip away that belief and get you to see that what is accepted as
mainstream economics both has evolved considerably over the past 250
years and continues to be contested to this day. While there is much
consensus, there continues to be much debate as to what economics
\emph{is}, what it is \emph{about}, what are the best \emph{methods} to
study it, what are the \emph{policy implications}, and what \emph{does
it tell us} about the world. The best way to understand these debates
and to take a position is to understand the history of economic thought:
how ideas of various thinkers and schools of thought were slowly
incorporated into the mainstream of economics, marginalized as
heterodox, or completely discarded; and how this process has changed
many times.

We will read famous texts by the major thinkers - Smith, Ricardo,
Malthus, Marx, Mill, Keynes - explore the key insights and development
of concepts and models from different schools of thought - mercantilist,
classical, Marxist, neoclassical, Austrian, institutional, Keynesian,
New- and Post-Keynesian, etc - and place developments and ideas into
historical context. As a one-semester course, we must inevitably leave
out many major thinkers, ideas, and debates in the history of economic
thought. If you have a particular interest that we do not cover,
consider this an ideal topic for a research paper.

Much of this course will examine the history of how modern economic
doctrine evolved, and this assumes that you have familiarity with modern
economic theory at the intermediate level. As such, the
\textbf{prerequisites} for this course are \textbf{ECON 305 -
Macroeconomic Analysis} \emph{and} \textbf{ECON 306 - Microeconomic
Analysis}.

As a \textbf{400-level course} required for all economics majors, this
course is intended to be an upper-level seminar to enrich your
understanding of economics beyond the theoretical tools you have already
learned. \textbf{This implies a significant amount of reading, writing,
and discussing, on your part, requiring you to take ownership of your
own learning.} Against my strong instincts to lecture, this class will
primarily be discussion-based and focused heavily on required readings
for each meeting.

\textbf{My standard disclaimer:} This class may challenge many of your
existing beliefs and conceptions about how the world works, and how it
should work. This is the most important and exciting part of a liberal
arts education. This does \emph{not} mean that I want to make you to see
everything ``my way.'' In fact, if you come out of this class thinking
exactly like me, then I have probably failed you as a teacher. To the
best of my ability, I keep my opinions to myself unless relevant for
purposes of discussion, and I respect and invite you to reach your own
conclusions on all matters.

If at any point you find yourself struggling in this course for any
reason, please come see me. Do not suffer in silence! Coming to see me
for help does not diminish my view of you, in fact I will hold you in
\emph{higher} regard for understanding your own needs and taking charge
of your own learning. There are also a some fantastic resources on
campus, such as the
\href{http://www.hood.edu/campus-services/academic-services/index.html}{Center
for Academic Achievement and Retention (CAAR)} and the
\href{http://www.hood.edu/library/}{Beneficial-Hodson Library}.

\clearpage

\hypertarget{hybrid-course-format}{%
\section*{Hybrid Course Format}\label{hybrid-course-format}}
\addcontentsline{toc}{section}{Hybrid Course Format}

This course is taught in a \textbf{hybrid} format, providing a mixture
of regular synchronous activity where we all can interact in real time,
with asynchronous material, which can be done remotely at your own pace.

\textbf{I will be holding all synchronous class sessions remotely (for
reasons I will make clear to you by the first day) on Zoom.} You can
attend these sessions on your computer or device from your dormitory or
home, and a classroom is available for you to use (socially-distanced,
and in masks), but I will not be in the classroom.

During the synchronous, scheduled times for the course (Monday/Wednesday
3:25 P.M.-4:45 P.M.), I will lecture on the material, hold in-class
discussions, and answer questions in real time \emph{on Zoom.}
Attendance to the live portion is strongly encouraged, but not required.

\textbf{Lecture slides, videos, and other synchronous materials will be
posted online by the end of the day in which the live session occurs.}

Assignments will always be submitted \emph{online} and due at regular
times (typically 11:59 PM Sunday) so that students unable to join in the
live sessions can complete them asynchronously.

Students are strongly encouraged to join the course
\href{https://hoodcollegeeconomics.slack.com}{Slack channel} to maintain
an active channel of communication, ask questions, and to build our
course community together. Official course-related announcements will
always come via Blackboard announcement and automatically sent to your
Hood email accounts.

\hypertarget{learning-in-a-time-of-coronavirus}{%
\subsection*{Learning in a Time of
Coronavirus}\label{learning-in-a-time-of-coronavirus}}
\addcontentsline{toc}{subsection}{Learning in a Time of Coronavirus}

Everything is awful right now. None of us signed up for this. None of us
are really okay, \textbf{we're all just pretending for everyone else.}

Many of you may be dealing with hardships at home and at work, and are
generally juggling many more problems than usual. Everyone's future
plans have been completely put on hold or cancelled to a large degree.
We all miss the sense of normalcy and human sense of community from
being isolated for so long.

For this unique semester, we are going to prioritize supporting each
other as human beings during this crazy era, and use simple, accessible
solutions that make sense for the most people, and above all, to be
flexible. I have designed the course to maintain some common structure
but be flexible to your varied needs. Please see the
\protect\hyperlink{policies-and-expectations}{policies and expectations
below}. I hope you use this course as an opportunity to escape the
boredom and insanity of social isolation, and to help keep interest in
understanding the world around us.

If you tell me you're having trouble, I will do whatever I can to help,
and not judge you or think less of you. I hope you will extend me the
same courtesy.

\hypertarget{course-objectives}{%
\section*{Course objectives}\label{course-objectives}}
\addcontentsline{toc}{section}{Course objectives}

\textbf{By the end of this course,} you will:

\begin{itemize}
\tightlist
\item
  Identify key economic writers and their contributions to economic
  thought
\item
  Discuss the influence and value of different writers and their
  contributions
\item
  Identify and distinguish the major schools of economic thought:
  Classical, Marxist, Neoclassical, Austrian, Keynesian, Monetarist, New
  Classical
\item
  Place theories and ideas studied within the context of the time
\item
  Better comprehend the origins and context of orthodox economic theory
  today
\item
  Explore critiques of orthodox economic theory from various heterodox
  schools of thought and understand contested areas
\item
  Reflect on the nature of economics as a discipline
\item
  Understand the origins of key economic concepts and models
\item
  Trace the evolution of major ideas through time, via your own writing
\end{itemize}

Given these objectives, this course fulfills two of the learning
outcomes for
\href{https://www.hood.edu/academics/departments/george-b-delaplaine-jr-school-business/student-learning-outcomes}{the
George B. Delaplaine, Jr.~School of Business} Economics B.A. program:

\begin{itemize}
\tightlist
\item
  Apply economic reasoning and models to understand and analyze problems
  of public policy {[}\ldots{]}
\item
  Demonstrate effective oral and written communications skills for
  personal and professional success{[}\ldots{]}
\end{itemize}

\hypertarget{required-course-materials}{%
\section*{Required Course materials}\label{required-course-materials}}
\addcontentsline{toc}{section}{Required Course materials}

This course requires regular online internet access. If you know you
will be unable to access the internet regularly, please let me know and
we can make arrangements.

You can find all course materials at my \textbf{dedicated website} for
this course:
\href{https://thoughtF20.classes.ryansafner.com}{thoughtF20.classes.ryansafner.com}.
Links to the website are posted on our Blackboard course page. Please
familiarize yourself with the website, see that it contains this
\href{https://thoughtF20.classes.ryansafner.com/syllabus/}{syllabus},
guides for your
\href{https://thoughtF20.classes.ryansafner.com/reference/}{reference},
and our
\href{https://thoughtF20.classes.ryansafner.com/schedule/}{schedule}. On
the schedule page, you can find each module with its own class page
(\textbf{start there!}) along with all related readings, lecture slides,
practice problems, and assignments.

My lecture slides will be shared with you, but you also have required
books necessary for completing assignments.

\hypertarget{books}{%
\subsection*{Books}\label{books}}
\addcontentsline{toc}{subsection}{Books}

Our readings are of two types: (1) \emph{primary sources} - the famous
texts of the writers themselves; and (2) a \emph{secondary textbook}
that will help you interpret the primary sources and place them in the
broader context of their day and in the evolution of economic thought.

\textbf{Primary Sources:}

\begin{enumerate}
\def\labelenumi{\arabic{enumi}.}
\tightlist
\item
  Medema, Steven G and Warren J Samuels, eds., 2013, \emph{The History
  of Economic Thought: A Reader}, 2\textsuperscript{nd} ed., New York:
  Routledge
\end{enumerate}

\textbf{Secondary Textbook}:

You must purchase a secondary textbook, but I will give you an option to
choose one. I will be drawing from both books in my lectures.

\begin{enumerate}
\def\labelenumi{\arabic{enumi}.}
\setcounter{enumi}{1}
\item
  Landreth, Harry and David C Colander, 2002,
  \href{https://www.amazon.com/History-Economic-Thought-Harry-Landreth/dp/0618133941}{\emph{History
  of Economic Thought}}, 4\textsuperscript{th} ed., Boston: Houghton
  Mifflin Company
\item
  Blaug, Mark, 1996, \emph{Economic Theory in Retrospect},
  5\textsuperscript{th} ed., New York: Cambridge University Press
\end{enumerate}

I intended on requiring Landreth and Colander (2002), but I was sad to
find that it is out of print. As such, I was unable to get it via Hood's
bookstore. There should be used copies on Amazon for a somewhat
reasonable price (it has been fluctuating between \$20-\$100). \emph{I
strongly recommend this book} because it is easier to read, focuses more
on historical context and big ideas than getting lost in the
math/models, and has more robust coverage of different schools of
thought (including heterodox) and extends right up to the present day.

Blaug (1996) is also an excellent book, but is more advanced, focusing
on fewer thinkers, and focusing in detail on their contributions to
modern economic \emph{theory} (i.e.~models and math) with very little
discussion of historical context. \emph{I recommend only getting Blaug
if you are unable to get Landreth and Colander for a reasonable price
(or at all).}

\emph{Do not purchase both books!}

\hypertarget{lecture-slides}{%
\subsection*{Lecture Slides}\label{lecture-slides}}
\addcontentsline{toc}{subsection}{Lecture Slides}

My lecture slides will be shared with you, serve as an additional
resource. My lectures will roughly follow the textbooks described above.

\hypertarget{articles}{%
\subsection*{Articles}\label{articles}}
\addcontentsline{toc}{subsection}{Articles}

Throughout the course, I will post both required and supplemental
(non-required) readings that enrich your understanding for each topic.
Check \emph{frequently} for announcements and updates to assignments,
readings, and grades.

\hypertarget{assignments-and-grades}{%
\section*{Assignments and Grades}\label{assignments-and-grades}}
\addcontentsline{toc}{section}{Assignments and Grades}

Your final course grade is the weighted average of the following
assignments. The assignments and grading for this course are summarized
in the table below, with descriptions of each below the table:

\begin{center}

\begin{tabular}{lll}
\toprule
 & Assignment & Percent\\
\midrule
n & Participation (Average) & 35\%\\
1 & Term Paper & 35\%\\
2 & Short Papers & 20\%\\
n & Tournament Votes & 10\%\\
\bottomrule
\end{tabular}
\end{center}

\hypertarget{short-papers}{%
\subsection*{Short Papers}\label{short-papers}}
\addcontentsline{toc}{subsection}{Short Papers}

There will be two short papers on assigned topics. The first will be
early in the semester, covering topics from the first several weeks. The
second will cover topics from the remaining weeks. I am looking for 3-5
page papers that explore subjects prompted by questions that I will
provide to you in advance. However, I also encourage you to take the
less-trodden path and explore something unique and interesting to you.

\hypertarget{term-paper}{%
\subsection*{Term Paper}\label{term-paper}}
\addcontentsline{toc}{subsection}{Term Paper}

Each of you will write a term paper tracing and critically discussing
the intellectual history of an economic topic of your choice. This
assignment will be scaffolded so that certain sections of the paper are
due partway through the semester so that I can provide feedback for you
to incorporate into the final draft. More information will be provided
in class.

\hypertarget{participation-and-discussion}{%
\subsection*{Participation and
Discussion}\label{participation-and-discussion}}
\addcontentsline{toc}{subsection}{Participation and Discussion}

Every week, we will have a discussion board thread on Blackboard. You
will be expected to contribute to the discussion board at least twice in
the week. Your weekly contribution will be graded out of 5 points. At
the end of the semester, I will apply the \emph{average} of your weekly
participation grades to apply (40\%) towards your final course grade.

I am interested in your thoughts, reactions, comments, and questions
about any of the material (lectures and/or readings). You do not need to
write more than a paragraph. Anything more than that, including
continuing to reply to each others' thoughts, questions, or comments,
(which I strongly hope you do!) is solely based on your own interest and
curiosity. I will jump in to answer questions the group is stuck on,
give my two cents, and stir the pot as needed. I strontly hope we still
keep a conversation going and can learn from each other, that was always
my goal, not to lecture at you! If you crave visual human contact, you
can submit your comments/reactions in the form of a short video, and we
can try that out! Though we might eventually need to move beyond
Blackboard in that case. We'll see how things go.

\hypertarget{tournament-votes}{%
\subsection{Tournament Votes}\label{tournament-votes}}

Since we all missed out on the NCAA Basketball Tournament last spring, I
have set up a tournament for ``the most interesting person in the
history of economics.'' Each era of economic thought (Classical,
Neoclassical, Heterodox, Modern) will have its own ``division'', in
which we will vote for the champion of each era (out of up to 8
thinkers), who will then go head to head against the champions of other
divisions for the title.

Once we complete an era, we will hold the votes in head-to-head matchups
between economists. Your assignment is to write up a short explanation
behind your vote (1-3 sentences) between each pair. This will allow you
to remember and place each thinker in the context of the history of
economic thought.

Whomever is able to most closely predict the outcome of the tournament
in advance will earn 20 bonus points.

\hypertarget{grading-scale}{%
\subsection*{Grading Scale}\label{grading-scale}}
\addcontentsline{toc}{subsection}{Grading Scale}

All grades are based on the following traditional scale:

\begin{center}

\begin{tabular}{llll}
\toprule
Grade & Range & Grade1 & Range1\\
\midrule
A & 93–100\% & C & 73–76\%\\
A− & 90–92\% & C− & 70–72\%\\
B+ & 87–89\% & D+ & 67–69\%\\
B & 83–86\% & D & 63–66\%\\
B− & 80–82\% & D− & 60–62\%\\
\addlinespace
C+ & 77–79\% & F & < 60\%\\
\bottomrule
\end{tabular}
\end{center}

See also my
\href{https://ryansafner.shinyapps.io/452_grade_calculator/}{
\texttt{Grade\ Calculator}} app where you can calculate your overall
grade using existing assignment grades and forecast ``what if''
scenarios.

These grades are firm cutoffs, but I do of course round upwards
(\(\geq 0.5\)) for final grades. A necessary reminder, as an academic, I
am not in the business of \emph{giving} out grades, I merely report the
grade that you \emph{earn}. I will not alter your grade unless you
provide a reasonable argument that I am in error (which does happen from
time to time).

\hypertarget{policies-and-expectations}{%
\section*{Policies and Expectations}\label{policies-and-expectations}}
\addcontentsline{toc}{section}{Policies and Expectations}

This syllabus is a contract between you, the student, and me, your
instructor. It has been carefully and deliberately thought out\footnote{A
  syllabus can and will be used as a legal document for disputes tried
  at a court of law. Ask me how I know.}, and I will uphold my end of
the agreement and expect you to uphold yours.

In the language of game theory, this syllabus is my commitment device. I
am a very understanding person, and I know that exceptions to rules
often need to be made for students. However, to be \emph{fair} to
\emph{all} students the syllabus artificially constrains my ability to
make exceptions at a whim for anyone. This prevents clever students from
exploiting my congenial personality at everyone else's expense. Please
read and familiarize yourself with the course policies and expectations
of you. Chances are, if you have a question, it is answered herein.

\hypertarget{online-attendance-and-participation}{%
\subsection*{Online Attendance and
Participation}\label{online-attendance-and-participation}}
\addcontentsline{toc}{subsection}{Online Attendance and Participation}

This is a hybrid course with synchronous parts. You are generally
expected to join (online via Zoom) our \textbf{synchronous} class
sessions unless circumstances prevent you from doing so. You are also
expected to log on at least weekly to review the week's course material
and post in the discussion board. Assignments are generally due by
11:59PM Sunday each week. Day-to-day attendance is not graded per se,
but your contributions to the course discussion constitute a
\emph{significant portion of your course grade.}

If you are unable to make a particular class, you generally do not need
to let me know. \textbf{The videos from all class sessions are posted on
Blackboard} so please review videos of classes you were unable to attend
live.

All assignments are able to be completed \textbf{asynchronously} during
the week, and are \textbf{generally due by 11:59PM Sunday each week} to
allow you flexibility in your hectic schedules.

\hypertarget{late-assignments}{%
\subsection*{Late Assignments}\label{late-assignments}}
\addcontentsline{toc}{subsection}{Late Assignments}

I will accept late assignments, but will subtract a specified amount of
points as a penalty. Even if it is the last week of the semester, I
encourage you to turn in late work: some points are better than no
points!

\hypertarget{netiquette}{%
\subsection*{Netiquette}\label{netiquette}}
\addcontentsline{toc}{subsection}{Netiquette}

When using Zoom and posting on the discussion board, please follow
appropriate internet etiquette (``Netiquette''). Written communications,
like blog posts or use of the Zoom chat, lacks important nonverbal cues
(such as body language, tone of voice, sarcasm, etc).

Above all else, please respect one another and think/reread carefully
about how others may see your post before you submit a comment. You are
expected to disagree and have different opinions, this is inherently
valuable in a discussion. Please be civil and constructive in responding
to others' comments: writing \emph{``have you considered `X'?''} is a
lot more helpful to all involved than just writing \emph{``well you're
just wrong.''}

Posting content that is wilfully incindiary, illegal, or that
constitutes academic dishonesty (such as plagarism) will automatically
earn a grade of 0 and may be elevated to other authorities on campus.

When using the chat function on Zoom, please treat it as official course
communications, even though I may not be grading it. It may be a quick
and informal tool - don't feel you need to worry about spelling or
perfect grammar - but please try to avoid \emph{too} informal
``text-speak'' (i.e.~say ``That's good for you'' instead of ``thas good
4 u'').

\hypertarget{late-assignments-1}{%
\subsection*{Late Assignments}\label{late-assignments-1}}
\addcontentsline{toc}{subsection}{Late Assignments}

I will accept late assignments, but will subtract a specified amount of
points as a penalty. See individual assignment descriptions for the
amount of points taken off (as it varies by assignment). If an answer
key is posted before you turn in your assignment, the maximum grade you
can earn is an 80. Even if it is the last week of the semester, I
encourage you to turn in late work: some points are better than no
points!

\hypertarget{grading}{%
\subsection*{Grading}\label{grading}}
\addcontentsline{toc}{subsection}{Grading}

I will try my best to post grades on Blackboard's Grading Center and
return graded assignments to you within about one week of you turning
them in. There will be exceptions. Where applicable, I will post answer
keys once I know most homeworks are turned in (see Late Assignments
above for penalties). Blackboard's Grading Center is the place to look
for your most up-to-date grades. You will also be given an Excel
spreadsheet template where you can calculate your overall grade and
forecast ``what if'' scenarios.

\hypertarget{communication-email-slack-and-virtual-office-hours}{%
\subsubsection*{Communication: Email, Slack, and Virtual Office
Hours}\label{communication-email-slack-and-virtual-office-hours}}
\addcontentsline{toc}{subsubsection}{Communication: Email, Slack, and
Virtual Office Hours}

Students must regularly monitor their \textbf{Hood email accounts} to
receive important college information, including messages related to
this class. Email through the Blackboard system is my main method of
communicating announcements and deadlines regarding your assignments.
\textbf{Please do not reply to any automated Blackboard emails - I may
not recieve it!}. My Hood email (\texttt{safner@hood.edu}) is the best
means of contacting me. I will do my best to respond within 24 hours. If
I do not reply within 48 hours, do not take it personally, and
\emph{feel free to send a follow up email} in the very likely event that
I genuinely did not see your original message.

Our \href{https://hoodcollegeeconomics.slack.com}{slack channel} is
available to all students and faculty in Economics and Business. I have
invited all of my classes and advisees. It will not be extended to
non-Business/Economics students or faculty. All users must use their
hood emails and true first and last names. Each course has its own
channel, exclusive for verified students in the course, and myself, by
my invite only. As a third party platform, you agree to its Terms of
Service. I have created this space as a way to stay connected, to help
one another, and to foster community. Behaviors such as posting
inappropriate content, harassing others, or engaging in academic
dishonesty, to be determined solely at my discretion, will result in one
warning, the content will be deleted, and subsequent behavior will
result in a ban.

I will host general \textbf{``office hours''} during my usual time,
3:30-5PM Mondays and Wednesdays, \href{https://zoom.us/j/458617463}{on
Zoom}. You can join in with video, audio, and/or chat, whichever you
feel comfortable with. Of course, if you are not available during those
times, we can schedule our own time if you prefer this method over email
or Slack. If you want to go over material from class, please have
\emph{specific} questions you want help with. I am not in the business
of giving private lectures (particularly if you missed class without a
valid excuse).

Watch the excellent and accurate video
\href{https://vimeo.com/270014784}{explaining office hours} (on website
syllabus page).

\hypertarget{enrollment}{%
\subsection*{Enrollment}\label{enrollment}}
\addcontentsline{toc}{subsection}{Enrollment}

Students are responsible for verifying their enrollment in this class.
The last day to add or drop this class with no penalty is \textbf{DATE}.
Be aware of
\href{https://www.hood.edu/offices-services/registrars-office/academic-calendar}{important
dates}.

\hypertarget{privacy}{%
\subsubsection*{Privacy}\label{privacy}}
\addcontentsline{toc}{subsubsection}{Privacy}

\href{https://www.execvision.io/blog/maryland-call-recording-laws/}{Maryland
law}
\href{https://law.justia.com/codes/maryland/2005/gcj/10-402.html}{requires}
all parties consent for a conversation or meeting to be recorded. If you
join in, and certainly if you participate, \textbf{you are consenting to
be recorded.} However, as described below, videos are not accessible
beyond our class.

Live lectures are recorded on Zoom and posted to Blackboard via Panopto,
a secure course management system for video. Among other nice features
(such as multiple video screens, close captioning, and time-stamped
search functions!), Panopto is authenticated via your Blackboard
credentials, ensuring that \emph{our course videos are not accessible to
the open internet.}

For the privacy of your peers, and to foster an environment of trust and
academic freedom to explore ideas, \textbf{do not record our course
lectures or discussions.} You are already getting my official copies.

The
\href{https://www2.ed.gov/policy/gen/guid/fpco/ferpa/index.html}{Family
Educational Rights and Privacy Act} prevents me from disclosing or
discussing any student information, including grades and records about
student performance. If the student is at least 18 years of age,
\emph{parents (or spouses) do not have a right to obtain this
information}, except with consent by the student.

Many of you may be tuning in remotely, living with parents, and may have
occasional interruptions due to sharing a space. This is normal and
fine, but know that I will protect your privacy and not discuss your
performance when parents (or anyone other than you, for that metter) are
present, without your explicit consent.

\hypertarget{using-slack}{%
\subsubsection*{Using Slack}\label{using-slack}}
\addcontentsline{toc}{subsubsection}{Using Slack}

Our \href{https://hoodcollegeeconomics.slack.com}{slack channel} is
available to all students and faculty in Economics and Business. I have
invited all of my classes and advisees. It will not be extended to
non-Business/Economics students or faculty. All users must use their
hood emails and true first and last names.

Each course has its own channel, exclusive for verified students in the
course, and myself, by my invite only.

As a third party platform, you agree to its Terms of Service.

I have created this space as a way to stay connected, to help one
another, and to foster community. Behaviors such as posting
inappropriate content, harassing others, or engaging in academic
dishonesty, to be determined solely at my discretion, will result in one
warning, the content will be deleted, and subsequent behavior will
result in a ban.

\hypertarget{honor-code}{%
\subsection*{Honor Code}\label{honor-code}}
\addcontentsline{toc}{subsection}{Honor Code}

Hood College has an Academic Honor Code which requires all members of
this community to maintain the highest standards of academic honesty and
integrity. Cheating, plagiarism, lying, and stealing are all prohibited.
All violations of the Honor Code are taken seriously, will be reported
to appropriate authority, and may result in severe penalties, including
expulsion from the college. See
\href{http://hood.smartcatalogiq.com/en/2016-2017/Catalog/The-Spirit-of-Hood/The-Academic-Honor-Code-and-Code-of-Conduct}{here}
for more detailed information.

\hypertarget{van-halen-and-mms}{%
\subsection*{Van Halen and M\&Ms}\label{van-halen-and-mms}}
\addcontentsline{toc}{subsection}{Van Halen and M\&Ms}

When you have completed reading the syllabus, email me a picture of the
band Van Halen and a picture of a bowl of M\&Ms.~If you do this
\emph{before} the date of the first exam, you will get bonus points on
the exam. If 75-100\% of the class does this, you each get 2 points. If
50-75\% of the class does this, you each get 4 points. If 25-50\% of the
class does this, you each get 6 points. If 0-25\% of the class does
this, you each get 8 points. Yes, this is real, and there is a story
behind this.

\hypertarget{accessibility-equity-and-accommodations}{%
\subsection*{Accessibility, Equity, and
Accommodations}\label{accessibility-equity-and-accommodations}}
\addcontentsline{toc}{subsection}{Accessibility, Equity, and
Accommodations}

College courses can, and should, be challenging and bring you out of
your comfort zone in a safe and equitable environment. If, however, you
feel at any point in the semester that certain assignments or aspects of
the course will be disproportionately uncomfortable or burdensome for
you due to any factor beyond your control, please come see me or email
me. I am a very understanding person and am happy to work out a solution
together. I reserve the right to modify and reweight assignments at my
sole discretion for students that I belive would legitimately be at a
disadvantage, through no fault of their own, to complete them as
described.

If you are unable to afford required textbooks or other resources for
any reason, come see me and we can find a solution that works for you.

This course is intended to be accessible for all students, including
those with mental, physical, or cognitive disabilities, illness,
injuries, impairments, or any other condition that tends to negatively
affect one's equal access to education. If at any point in the term, you
find yourself not able to fully access the space, content, and
experience of this course, you are welcome to contact me to discuss your
specific needs. I also encourage you to contact the
\href{https://www.hood.edu/academics/josephine-steiner-center-academic-achievement-retention/accessibility-services}{Office
of Accessibility Services} (301-696-3421). If you have a diagnosis or
history of accommodations in high school or previous postsecondary
institutions, Accessibility Services can help you document your needs
and create an accommodation plan. By making a plan through Accessibility
Services, you can ensure appropriate accommodations without disclosing
your condition or diagnosis to course instructors.

\hypertarget{tentative-schedule}{%
\section*{Tentative Schedule}\label{tentative-schedule}}
\addcontentsline{toc}{section}{Tentative Schedule}

Below is a rough sketch of the weekly schedule we will aim to follow
this semester. Each module should take approximately one class meeting.

\textbf{You can find a full schedule} with much more details, including
the readings, appendices, and other further resources for each class
meeting on the
\href{http://thoughtF20.classes.ryansafner.com/schedule/}{course
website's schedule page}.

\begin{center}
\small

\begin{tabular}{llll}
\toprule
Week & Topics & Readings & Assignments\\
\midrule
8/16-8/22 & Introduction & Reader Ch.1 & \\
 & Key Themes & Reader Ch.1 & \\
8/23-8/29 & Ancient Greeks & Reader Ch.2 & \\
 & Medieval Writers & Reader Ch.3 & \\
8/30-9/5 & Physiocrats & Reader Ch.4 & \\
\addlinespace
 & Mercantilists & Reader Ch.5 & \\
9/6-9/12 & Pre-Smithian Political Economy & Reader Ch.6 & \\
 & The Invisible Hand & Reader Ch.7 & \\
9/13-9/19 & Malthus \& Population & Reader Ch.8 & \\
 & The Ricardian System & Reader Ch.9 & \\
\addlinespace
9/20-9/26 & Say \& Classical 'Macroeconomics' & Reader Ch.10 & \\
 & The Marxist Challenge & Reader Ch.11 & \\
9/27-10/3 & Mill \& The Apex of Classical Economics & Reader Ch.12 & Paper 1 Due\\
 & The Marginalist Revolution & Reader Ch.13 & \\
10/4-10/10 & Marshallian Demand \& Utility & Reader Ch.14 & \\
\addlinespace
 & Marshallian Theory of the Firm & Reader Ch.14 & \\
10/11-10/17 & Marginal Productivity Theory & Reader Ch.15 & \\
 & Pareto, Pigou, \& Welfare Economics & Reader Ch.15 & \\
10/18-10/24 & Austrians \& Capital Theory & Reader Ch.16 & \\
 & Socialist Calculation Debate & Reader Ch.17 & \\
\addlinespace
10/25-10/31 & American Institutionalist Economics & Reader Ch.18 & Paper 2 Due\\
 & Mathematicization \& Samuelsonian Economics & Reader Ch.19 & \\
11/1-11/7 & Pre-Keynesian Macroeconomics & Reader Ch.20 & \\
 & The Keynesian Revolution & Reader Ch.20 & \\
11/8-11/14 & Monetarism & Reader Ch.21 & \\
\addlinespace
 & New Classical Macroeconomics & Reader Ch.22 & \\
11/15-11/21 & Public Choice & Reader Ch.23 & \\
 & New Institutional Economics & Reader Ch.24 & \\
11/22-11/28 & Econometrics & Reader Ch.25 & Term Paper Due\\
 & Economics Today & Reader Ch.26 & \\
\bottomrule
\end{tabular}
\end{center}

\end{document}